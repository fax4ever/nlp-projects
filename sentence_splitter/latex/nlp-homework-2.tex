\pdfoutput=1
\documentclass[11pt]{article}

\usepackage{acl}
\usepackage{times}
\usepackage{latexsym}
\usepackage[T1]{fontenc}
\usepackage[utf8]{inputenc}
\usepackage{microtype}
\usepackage{inconsolata}
\usepackage{graphicx}
\usepackage{booktabs}
\usepackage{numprint}
\usepackage{listings}

\title{%
	Sentence Splitter\\
	\large Multilingual Natural Language Processing \\
	Homework 2}
\date{August $10^{th}$, 2025}

\author{Ercoli Fabio Massimo \\
\texttt{802397} \\\And
Della Porta Nicolò \\
\texttt{1920468} \\}

\begin{document}

\maketitle

\section{Introduction}

\section{Methodology}

explain your starting hypothesis and the methods you will use to test them

\subsection{Embedding-based}

\subsection{LLM-based}

\section{Experiments}

explain your experimental setup; here you give technical details on
models, resources, etc. and how you implemented what you described before

Test the model(s) on the OOD.
Building a sentence splitter that actually works Out-of-Domain (ODD) is a non-trivial task in NLP.

\subsection{Results}

present the results referencing tables and commenting them, i.e., why you
accept or refuse your starting hypothesis and how you explain the (most
notable/unexpected) results.

Comparative analysis of the two approaches on the validation set – which one is
better?

\appendix
\section{Training results Appendix}
\label{sec:appendix1}

\end{document}

