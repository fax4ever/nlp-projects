\pdfoutput=1
\documentclass[11pt]{article}

\usepackage{acl}
\usepackage{times}
\usepackage{latexsym}
\usepackage[T1]{fontenc}
\usepackage[utf8]{inputenc}
\usepackage{microtype}
\usepackage{inconsolata}
\usepackage{graphicx}
\usepackage{booktabs}
\usepackage{numprint}
\usepackage{listings}
\usepackage{natbib}
\bibliographystyle{unsrtnat}

\title{%
	Sentence Splitter\\
	\large Multilingual Natural Language Processing \\
	Homework 2}
\date{August $10^{th}$, 2025}

\author{Ercoli Fabio Massimo \\
\texttt{802397} \\\And
Della Porta Nicolò \\
\texttt{1920468} \\}

\begin{document}

\maketitle

\section{Introduction}

Quoting \cite{redaelli-sprugnoli-2024-sentence} "Sentence splitting, that is the segmentation of the raw input text into sentences, is a fundamental step in text processing".  According to \cite{frohmann2024segmenttextuniversalapproach} the main challenges are:

\begin{itemize}
	\item robustness to missing punctuation
	\item effective adaptability to new domains
	\item high efficiency
\end{itemize}

According to \cite{redaelli-sprugnoli-2024-sentence} we can add to the list:

\begin{itemize}
	\item multilinguality
\end{itemize}

Because a sentence splitting that works well for English may not work well to split another language.

In this project we implemented two models for sentence splitting, using an Italian corpus as train and validation set.
The first one is based on an embedding model, the second one is based on generative LLM.
We want to analyze compare the two approaches. 
We want also to test the models out of domains.

\section{Methodology}

\subsection{Embedding-based}

We fine tuned a pretrained embedding model (multilingual or trained on Italian language) using the test and validation datasets provided by the homework guide.

The original dataset provides two large texts (one to use as train and the other as validation) together with the golden labels to mark the end of sentence (1) and all the rest of the words (0).

In order to make the datasets suitable for the training we had to apply some transormations. 
We needed to group words into sequences of tokens aligning the golden labels consistently.
The number of tokens of each sequence must fit the max length of the embedding models,
for instance 512.

We generated different datasets using different number of words to generate each sequence: 
64, 128, 192, 256. Notice that the number of tokens for each sequences will be strictly grater,
since for each word the tokenizer of the model will produce one or more tokens.

Before to use the datasets we still need to align (as mentioned above)  the labels to the tokens.
The alignment strategy we applied consists in keep 1 as the first token generated from a word
having label 1, use 0 for all the other cases.

One aspect that we had to address was the fact that the label distribution is very unbalanced
for this use case. Most the labels are 0s, and few 1s. We applied 2 different ideas.

First of all, we set the \emph{load\_best\_model\_at\_end} as training argument to \emph{true},
using \emph{metric\_for\_best\_model} to F1. This to implement an early stopping based on F1,
and not on the accuracy (that would be misleading for such a unbalanced dataset).

Second of all,  we override the loss function to weight much less (1/30) a miss-labeling on 0s then
the ones on 1s. We called it a weighted trainer. 

\subsection{LLM-based}

\section{Experiments}

explain your experimental setup; here you give technical details on
models, resources, etc. and how you implemented what you described before

Test the model(s) on the OOD.
Building a sentence splitter that actually works Out-of-Domain (ODD) is a non-trivial task in NLP.

\subsection{Results}

present the results referencing tables and commenting them, i.e., why you
accept or refuse your starting hypothesis and how you explain the (most
notable/unexpected) results.

Comparative analysis of the two approaches on the validation set – which one is
better?

The \LaTeX\ related items are
\cite{frohmann2024segmenttextuniversalapproach}. 

The \LaTeX\ related items are
\cite{redaelli-sprugnoli-2024-sentence}. 

\appendix
\section{Training results Appendix}
\label{sec:appendix1}

\bibliography{custom}

\end{document}

