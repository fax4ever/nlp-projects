\pdfoutput=1
\documentclass[11pt]{article}

\usepackage{acl}
\usepackage{times}
\usepackage{latexsym}
\usepackage[T1]{fontenc}
\usepackage[utf8]{inputenc}
\usepackage{microtype}
\usepackage{inconsolata}
\usepackage{graphicx}

\title{Report MNLP}
%\date{May $4^{th}$, 2025}

\author{Ercoli Fabio Massimo \\
\texttt{802397} \\\And
Della Porta Nicolò \\
\texttt{1920468} \\\And
Regina Giovanni \\
\texttt{1972467} \\}

\begin{document}

	\maketitle

	\section{Introduction}
	Cultural items are elements such as concepts or entities that carry cultural meaning and reflect the identity, practices, and values of specific communities. In natural language, these items can appear in diverse forms, ranging from food names and historical references to gestures and works of art. Their interpretation often depends on shared knowledge within a culture, making their automatic classification a complex task.
	In this report, it is described how we addressed the task of automatic cultural item classification. The goal is to label each item by identifying the category it belongs to among the three given categories: \textit{Cultural Agnostic (CA)}, \textit{Cultural Representative (CR)} and \textit{Cultural Exclusive (CE)}. As requested, to tackle this we implemented and evaluated two distinct approaches: a LM-based method using an encoder Transformer and a non-LM-based method relying on several data. The report presents a comparative analysis of the two approaches in terms of classification performance and it explains how they work by reflecting on the methodological choices employed.

	\section{Methodology}

	\subsection{Non-LM-based}
	extraction of information from Wikipedia and Wikidata pages
	\subsection{LM-based}

	\section{Experiments}

	\section{Results}


	\appendix

	\section{Example Appendix}
	\label{sec:appendix}

	This is an appendix.

\end{document}

